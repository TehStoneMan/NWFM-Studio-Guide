% Options for packages loaded elsewhere
\PassOptionsToPackage{unicode}{hyperref}
\PassOptionsToPackage{hyphens}{url}
%
\documentclass[
]{article}
\usepackage{amsmath,amssymb}
\usepackage{lmodern}
\usepackage{iftex}
\ifPDFTeX
  \usepackage[T1]{fontenc}
  \usepackage[utf8]{inputenc}
  \usepackage{textcomp} % provide euro and other symbols
\else % if luatex or xetex
  \usepackage{unicode-math}
  \defaultfontfeatures{Scale=MatchLowercase}
  \defaultfontfeatures[\rmfamily]{Ligatures=TeX,Scale=1}
\fi
% Use upquote if available, for straight quotes in verbatim environments
\IfFileExists{upquote.sty}{\usepackage{upquote}}{}
\IfFileExists{microtype.sty}{% use microtype if available
  \usepackage[]{microtype}
  \UseMicrotypeSet[protrusion]{basicmath} % disable protrusion for tt fonts
}{}
\makeatletter
\@ifundefined{KOMAClassName}{% if non-KOMA class
  \IfFileExists{parskip.sty}{%
    \usepackage{parskip}
  }{% else
    \setlength{\parindent}{0pt}
    \setlength{\parskip}{6pt plus 2pt minus 1pt}}
}{% if KOMA class
  \KOMAoptions{parskip=half}}
\makeatother
\usepackage{xcolor}
\usepackage{longtable,booktabs,array}
\usepackage{calc} % for calculating minipage widths
% Correct order of tables after \paragraph or \subparagraph
\usepackage{etoolbox}
\makeatletter
\patchcmd\longtable{\par}{\if@noskipsec\mbox{}\fi\par}{}{}
\makeatother
% Allow footnotes in longtable head/foot
\IfFileExists{footnotehyper.sty}{\usepackage{footnotehyper}}{\usepackage{footnote}}
\makesavenoteenv{longtable}
\setlength{\emergencystretch}{3em} % prevent overfull lines
\providecommand{\tightlist}{%
  \setlength{\itemsep}{0pt}\setlength{\parskip}{0pt}}
\setcounter{secnumdepth}{-\maxdimen} % remove section numbering
\ifLuaTeX
  \usepackage{selnolig}  % disable illegal ligatures
\fi
\IfFileExists{bookmark.sty}{\usepackage{bookmark}}{\usepackage{hyperref}}
\IfFileExists{xurl.sty}{\usepackage{xurl}}{} % add URL line breaks if available
\urlstyle{same} % disable monospaced font for URLs
\hypersetup{
  hidelinks,
  pdfcreator={LaTeX via pandoc}}

\author{}
\date{}

\begin{document}

\hypertarget{media}{%
\section{Media}\label{media}}

\hypertarget{audio-file-formats}{%
\subsection{Audio file formats}\label{audio-file-formats}}

Audio files have been stored in three different formats. The following
table will show the main differences between the formats used at the
station.

\begin{longtable}[]{@{}
  >{\raggedright\arraybackslash}p{(\columnwidth - 8\tabcolsep) * \real{0.1415}}
  >{\raggedright\arraybackslash}p{(\columnwidth - 8\tabcolsep) * \real{0.2584}}
  >{\raggedright\arraybackslash}p{(\columnwidth - 8\tabcolsep) * \real{0.2000}}
  >{\raggedright\arraybackslash}p{(\columnwidth - 8\tabcolsep) * \real{0.2000}}
  >{\raggedright\arraybackslash}p{(\columnwidth - 8\tabcolsep) * \real{0.2001}}@{}}
\toprule()
\begin{minipage}[b]{\linewidth}\raggedright
Extension
\end{minipage} & \begin{minipage}[b]{\linewidth}\raggedright
Compression
\end{minipage} & \begin{minipage}[b]{\linewidth}\raggedright
Lossiness
\end{minipage} & \begin{minipage}[b]{\linewidth}\raggedright
Quality
\end{minipage} & \begin{minipage}[b]{\linewidth}\raggedright
File size
\end{minipage} \\
\midrule()
\endhead
.flac & Compressed data file & Lossless format & Full quality & Medium
file size. \\
.mp3 & Compressed audio file & Lossy format & Variable quality &
Smallest file size \\
.wav & Uncompressed file & Lossless format & Full quality & Largest file
size \\
\bottomrule()
\end{longtable}

Note the difference between compressed audio and compressed data. With
compressed audio, the audio information itself is altered to provide a
smaller file size, but it results in a loss of audio quality. In
contrast, compressed data means that while the data that makes up the
audio is compressed, when the data is converted back into audio by the
computer, the quality of the audio is unaffected.

It is therefore recommended to use the .flac file format for final
playout audio where possible as it offers the best compromise between
file size and audio quality. Audio editing should always be performed
using .wav format files.

.mp3 files should only be used if that is how the file was supplied.

\end{document}
