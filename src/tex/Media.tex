\documentclass[../StudioOperationGuide.tex]{subfiles}
\begin{document}

\chapter{Media}
TODO: Describe differences in audio file formats.
%\hypertarget{media}{%
%\section{Media}\label{media}}

%\hypertarget{audio-file-formats}{%
%\subsection{Audio file formats}\label{audio-file-formats}}

Audio files have been stored in three different formats. The following
table will show the main differences between the formats used at the
station.

%\begin{longtable}[]{@{}
%  >{\raggedright\arraybackslash}p{(\columnwidth - 8\tabcolsep) * \real{0.1415}}
%  >{\raggedright\arraybackslash}p{(\columnwidth - 8\tabcolsep) * \real{0.2584}}
%  >{\raggedright\arraybackslash}p{(\columnwidth - 8\tabcolsep) * \real{0.2000}}
%  >{\raggedright\arraybackslash}p{(\columnwidth - 8\tabcolsep) * \real{0.2000}}
%  >{\raggedright\arraybackslash}p{(\columnwidth - 8\tabcolsep) * \real{0.2001}}@{}}
%\toprule()
%\begin{minipage}[b]{\linewidth}\raggedright
%Extension
%\end{minipage} & \begin{minipage}[b]{\linewidth}\raggedright
%Compression
%\end{minipage} & \begin{minipage}[b]{\linewidth}\raggedright
%Lossiness
%\end{minipage} & \begin{minipage}[b]{\linewidth}\raggedright
%Quality
%\end{minipage} & \begin{minipage}[b]{\linewidth}\raggedright
%File size
%\end{minipage} \\
%\midrule()
%\endhead
%.flac & Compressed data file & Lossless format & Full quality & Medium
%file size. \\
%.mp3 & Compressed audio file & Lossy format & Variable quality &
%Smallest file size \\
%.wav & Uncompressed file & Lossless format & Full quality & Largest file
%size \\
%\bottomrule()
%\end{longtable}

Note the difference between compressed audio and compressed data. With
compressed audio, the audio information itself is altered to provide a
smaller file size, but it results in a loss of audio quality. In
contrast, compressed data means that while the data that makes up the
audio is compressed, when the data is converted back into audio by the
computer, the quality of the audio is unaffected.

It is therefore recommended to use the .flac file format for final
playout audio where possible as it offers the best compromise between
file size and audio quality. Audio editing should always be performed
using .wav format files.

.mp3 files should only be used if that is how the file was supplied.

\end{document}
